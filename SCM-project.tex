

\documentclass[10pt]{article}

% amsmath package, useful for mathematical formulas
\usepackage{amsmath}
% amssymb package, useful for mathematical symbols
\usepackage{amssymb}

% graphicx package, useful for including eps and pdf graphics include graphics with the command \includegraphics
\usepackage{graphicx}

% Tells latex that the images are kept in a folder named figures under the current directory. 
\graphicspath{ {figures/} }

% cite package, to clean up citations in the main text. Do not remove.
\usepackage{cite}

\usepackage{color} 

% inputenc package, allows the user to input accented characters directly from the keyboard
\usepackage[utf8]{inputenc}

\usepackage[parfill]{parskip}

% hyphenat package, can be used to disable all hyphenation in a document or in selected text within the document
\usepackage[none]{hyphenat}

% The package hyperref provides LaTeX the ability to create hyperlinks within the document.
\usepackage{hyperref}

% tocbibind package, automatically adds the bibliography and/or the index and/or the contents etc.., to the Table of content listing. 
\usepackage[nottoc,notlot,notlof]{tocbibind}

\usepackage{titlesec}
\titlespacing{\paragraph}{15pt}{*4}{*1.5}

%Abstract support /JOKL
%\renewcommand\abstractname{\textit{Abstract}}
\renewenvironment{abstract}
  {\small\quotation
  {\bfseries\noindent{\large\abstractname}\par\nobreak\smallskip}}
  {\endquotation}
  
\providecommand{\keywords}[1]{\textbf{{Keywords---}} #1}

\newcommand\remove[1]{\textcolor{red}{#1}}

%Sets the length of indents when beginning on a new line in a paragraph. 
\setlength{\parindent}{1cm}

% The todonotes package allows you to insert to{do items in your docu-ment. At any point in the document a list of all the inserted to{do items can be listed with the \listoftodos command
\usepackage{todonotes}

\usepackage[toc,page]{appendix}

%\titlespacing\section{0pt}{12pt plus 4pt minus 2pt}{0pt plus 2pt minus 2pt}
\setcounter{secnumdepth}{5}
\setcounter{tocdepth}{5}

% Use doublespacing - comment out for single spacing
\usepackage{setspace} 

% Text layout
\topmargin 0.0cm
\oddsidemargin 0.5cm
\evensidemargin 0.5cm
\textwidth 16cm 
\textheight 21cm

% Bold the 'Figure #' in the caption and separate it with a period
% Captions will be left justified
\usepackage[labelfont=bf,labelsep=period,justification=raggedright]{caption}
\captionsetup[table]{name=Tabell}
% Use the PLoS provided bibtex style
\bibliographystyle{plos2009}

%\newcommand\todo[1]{\textcolor{red}{#1}}
\begin{document}

% Remove brackets from numbering in List of References
\makeatletter
\renewcommand{\@biblabel}[1]{\quad#1.}
\makeatother


\pagestyle{myheadings}


% Title must be 150 characters or less
\begin{titlepage}
\title{Introduction to Software Configuration Management}
\author{Carl-Johan Heinze, Jonathan Klingberg, Jerry Lindström \\Group II-f}
\date{\today}
\maketitle
\thispagestyle{empty}
\end{titlepage}

%\newpage
\tableofcontents
\thispagestyle{empty}
\newpage
\pagenumbering{arabic}

%%%%% Requirement engineering work %%%%%%%%% 
\begin{abstract}
ca 5 rader per man.
\end{abstract}
\keywords{Software Configuration Management (SCM), parallel development, version control, team coordination, SCM tool evaluation, agile development, return of investment, configuration items, SCM implementation plan, CM variants, branching patterns.}

\section{Introduction}
This report is about helping a small but growing company with a low budget and without previous SCM experience to adopt SCM. The company today suffers from the common problems that a proper SCM implementation would solve. What is relevant is how to convince the company that SCM is needed and that it would save them time, effort and money. 

\noindent Further we will focus on some of the implementation tasks found in Moreiras \cite{Moreira} paper about SCM implementation such as how to raise management commitment of SCM, setup viable branching patterns and how to select a SCM tool suitable for the company in order to increase their productivity. During this process we will address some of the common pitfalls and provide a sustainable solution where both current and future circumstances are taken into account.

\section{PonteVecchio Software Biography}
The company which is target for this project is called PonteVecchio Software. PonteVecchio has grown from 2 to 15 developers over the last 10 years, and has never used any type of versioning or configuration management. Their software has been backed up to CDs each time it has been sent to a customer, but they do not support the older versions, or any other version for that matter.

\noindent PonteVecchio Software started out as a small software firm with two Computer Science students as their founders, John and Adam. Their first projects was a small back end service for a web server which they sold directly to other developers. The service was a moderate success and motivated John and Adam to hire some more people and take on new challenges. Now 10 years later, PonteVecchio has 4 products which their 15 developers ship to customers:
\begin{itemize}
\item A task manager app for keeping schedules and todo list. This app is developed in two versions, both as a regular desktop application and a mobile application. Both share a high degree of code base. 
\item An online password manager where users can store all their login information online. This system requires a high level of security verification.
\item \remove{An enterprise version of their scheduling app that was specifically ordered by a company. This version is highly customized and does not share a lot of code base with the standard application. It does however use some Open Source developed modules that the company requires for integration with their systems.} \todo{Vi har väl tillräckligt med komplexitet redan, detta känns irrelevant?} 
\end{itemize}

\noindent Since the enterprise customers are very demanding and comes up with new modifications and requirements, the developers have chosen to work agile and uses monthly scrum sessions. In the end of each scrum they will do a new release which is evaluated by the customers and developed further in the next sprint.

\noindent Up until now all projects and its modules has been shared either via a local server in the office or via email, with each developer being responsible for their own share. This has severely hurt their productivity since the possibility of concurrent development is very limited. 

\noindent Today John is the one that knows the organization best and has tried to keep up with what is happening in every project and what the others are currently working on. John is also the one with most panic to get the SCM in place before something catastrophic could happen. Therefore John is eager on taking the role of a CM manager. John is the one who suggested taking on SCM, even though he himself has very limited knowledge of SCM and how it works.

\noindent Adam often do experiments on new features and wants to try them out before implementing them to the main repository, he therefore would like to have his own environment to do his experiments on. PonteVecchio has a steady revenue from these product but their margins are slim and they need to keep costs down, therefore John also desires a system that is highly cost-effective to implement, both in terms of price and time effort. 

\noindent The Quality Assurance or QA are currently low on resources and more or less all of the developers acts as QA. Hence it’s not uncommon that errors are delivered to the customer and since some parts of the system are security critical the traceability of changes made between releases are as important as the possibility to rollback and reproduce a stable version in this scenario.

\noindent The company's future looks bright and is about to hire new personnel in a near future but wants their SCM implementation to be in place before new people gets involved. Therefore the SCM setup we will implement must be future compatible with more people along with separated departments and teams for the different projects. Further the responsibility for the different modules must be easy to split between the developers.

\section{Problem statement}

\hfill \break
\noindent Some of the problems of this report are  listed as follows:
\begin{itemize}
\item How are we going to convince the stakeholders to adopt SCM if they have survived almost completely without it so far? In other words what are the greatest benefits of SCM?
\item How will it be possible to introduce SCM if the company is already on a tight budget?
\item What are the difficulties in order to implement SCM to a company new to SCM?
\item \remove{Are additional people needed in the company to handle SCM and what properties are we looking for when choosing someone to be responsible for SCM?}\todo{Vi nämner inte detta i rapporten?} 
\item Which branching pattern, merging strategy and \remove{synchronization model}\todo{Detta saknas? Ska vi ta bort eller skriva något om det?} will be the most suitable for PonteVecchio considered the high demand for traceability along with the complexity of parallel development and variants?
\item What needs to ba taken into account when selecing a SCM tool suitable for PonteVecchio considered the high demand for traceability and security verification within a reasonable budget?
\item \remove{How can we help the company to maintain a sustainable SCM after the basic implementation is finished?}\todo{Vi nämner inte detta i rapporten? Lars tyckte denna punkt var intressant, lite synd att vi inte nämner något...} 
\todo{Kan vi härleda alla problem vi diskuterar i rapporten till dessa nu? eller saknas något?}
\end{itemize}

\section{Method}
First of all we’ll need to convince the importance of SCM to the management of PonteVecchio, this can be done for example by highlighting some of their current problems that a proper SCM implementation would solve, we can also refer so statistics from similar companies with successfully implemented SCM.

\noindent Since PonteVecchio are completely unfamiliar to SCM we’ll need to setup a SCM process including branching pattern and merging strategies tailored for their needs. We’ll also describe the most fundamental tasks of the SCM tool evaluation process which will help them to find a tool suitable for their needs.

\noindent \remove{The implementation plan will also include steps such as setting up a SCM policy and some guidelines for how SCM should be applied, this to make sure the SCM structure are being followed even after the implementation is finished.}

\noindent \remove{The planned preliminary end result is a more organized, effective and well structured organization. The cost of having SCM shall be justified by having less problems. Overall it shall be beneficial to have SCM in the company, compared to not having it.}

\section{Scope}
Moreira mentions in \cite{Moreira} that in order to increase the chances of successfully implement SCM there's 4 common success criterias for a SCM implementation effort that needs to be fulfilled on a reasonable level before the implementation can be carried out:
\begin{itemize}
\item Funding - money to purchase appropriate SCM tools ans infrastructure
\item Skilled SCM personnel - persons trained and experienced in the areas of SCM tools and process
\item Sponsorship - Management commitment to the effort
\item SCM implementation plan - a plan detailing the tasks that lead to the implementation of SCM practices and effective tracking and management of the plan
\end{itemize}
Of which we will concentrate on sponsorship and the implementation plan. Our intention is not to cover a complete implementation plan, but rather to discuss some the most fundamental parts of it such as branching patterns and selection of SCM tools. The rest does not fit into the scope of this report.
 
\section{Convince management (Ny rubrik) /CJ}
The level of configuration management at PonteVecchio has up until now been at a very primitive level, and is in effect hindering the company from developing, maintaining and expanding. In order to convince the management that structuring an SCM plan and employing SCM tools is the right way to go instead of using their ad hoc solution, we would like to introduce the Capability Maturity Model (CMM). 

\noindent CMM is a model developed by the Software Engineering Institute (SEI) and is used as an example of how costs may be lowered and benifits increased by maturing software projects through enacting formally defined processes. According to the CMM, implementing SCM is a level 2 key process, where level 1 is remaining at an ad hoc process.

\noindent An SEI report \cite{Merant} from 1992 averaged data from 1,233 separate software projects in 261 organizations and 10 countries in order to validate the benefits of increasing the maturity of a projects.
\begin{table}[h!]
    \centering
        \begin{tabular}{| p{2cm} | p{2cm} | p{2cm} | l | l | r |}
         \hline
        Maturity Level & Calendar Months & Effort (Work Months) & Defects Found & Defects Shipped & Total Cost \\ \hline
        1 & 29.8 & 593.5 & 1348 & 61 & \$5,440,00 \\ \hline
        2 & 18.5 & 143.0 & 328 & 12 & \$1,311,00 \\ \hline
        3 & 15.2 & 79.5 & 182 & 7 & \$728,00 \\ \hline
        4 & 12.5 & 42.8 & 97 & 5 & \$392,00 \\ \hline
        5 & 9.0 & 16.0 & 37 & 1 & \$146,00 \\ \hline
    \end{tabular}
    \caption{CMM Maturity Benefits}
    \label{CMM}
\end{table}

\noindent It should be mentioned that this report is over 20 years old, however it still clearly illustrates the benefit of having a clearly defined process for configuration management. 

\noindent However, this model only addresses the difference in costs depending on the projects maturity level. Moving from one level to another has inherent costs involved. These costs involve both cost for tools, educating developers in the tools, training managers for handling the systems , etc. In order to convince the stakeholders that the costs will be worth the investments we should also make a presentation of the return on investment (ROI) for implementing SCM.

\noindent Borraci\cite{Borraci}, in his master thesis, suggests one way for calculating the ROI with the goal of anwering the question "How much?" in regards to the implementation of SCM. By grouping the different benefits and costs generated by measurable, partially measureable and non measurable data, Borraci reached a formula for calculating the ROI of implementing SCM.

\section{Difficulties implementing SCM in companies that are new to SCM /JL}
This subsection will not go too deep on the above-mentioned difficulties, as these are mentioned in each relevant subsection that belongs to each particular \remove{SCM task}.

Some difficulties worth highlightning are the following:
\begin{itemize}
\item How are we going to get the developers involved in defining SCM plans and activities for their current project? It is not uncommon for developers that have not used SCM in the past, to be disinterested and think it is a waste of time. In general, the easiest way to make people work on something that they are reluctant to work on, is either to force them to do it, or to motivate them to do it. Therefore, we recommend having a workshop \cite{Vinter}, as it would indirectly force the developers to sit down and think about how SCM can benefit their current project. 

\noindent{}We all know that human nature is such that when we see the benefits of things in general, we will be motivated to work in order to gain these benefits. In the long run, the workshop will give the developers a taste of the benefits of SCM, which will make them more self motivated to conduct SCM activities on their own. 

\item Another difficulty worth highlightning, is how to go through all existing artifacts and choose which ones shall be configuration items (CI:s) \cite{Kelly2}? It is obvious that we need to identify what items require individual configuration, which is not an easy task, especially for a company without prior SCM knowledge. 

\noindent{}Assuming that we do not have anyone in the company who is familiar with SCM, the main debate is, will we dare to take the risk to choose the CI:s on our own, or will it be more beneficial to hire an expert to analyze our system and determine what we shall consider to be CI:s? In these situations, we recommend external consulting, as choosing the right CI:s can be rather difficult \cite{Daniels3}\cite{Kelly2}.

\item A third difficulty worth highlighting, is how to ensure reproducibility of everything prior to SCM? The answer is that this is an extremely hard task, especially if we want to reproduce exact versions of source code that a certain customer is using \cite{Appleton2}.

\noindent{}The main debate here is if SCM have been introduced too late, which means loss of traceability \cite{Kelly} (which indirectly affects reproducibility, because one must be able to trace what files or functionality were present in each version). Unless the company already had good reproducibility or traceability prior to SCM, this may be an impossible task \cite{Bays}. 

\end{itemize}
\section{Branching pattern/JL}
This subsection covers some important branching related topics. Some topics are more detailed than others, depending on how much emphasis we have chosen to put on each specific subject.

\subsection{Optimal branching for Parallel Development/JL}
The main question here is how branching should be performed in order to obtain maximum efficiency during parallel work.
\todo{Försök gärna göra lite underrubriker så blir det lite mer lättöverskådligt}

\noindent Our definition of not having any branches would be that no branches are allowed, not even local ones in the workspace. This means that everyone would be working on the original files in the repository. Parallel work would be very difficult in such a scenario, as everyone would be intruding into each others space, braking each others code \cite{Appleton}.

\noindent Not having any branches would mean bad integrity and it would be more difficult to track, verify and reproduce configurations and features \cite{Appleton}. Since reproducibility is harder to ensure without branches, this leads to bad security. With bad security, we mean that we can not go back to a previous version of our released software. This can be problematic because if our new revision have more bugs than our old \cite{Babich}, we can't do a diff between the new and the old version to find the bug, because we did not branch the old version. Reproducibility would be nonexistent without any branches, as the only version we have is the one we are working on.

\noindent It is obvious that branches are needed in order to work efficiently, both in terms of parallel work and after release product support. How is our branching strategy supposed to be in order to reach optimum results in these fields?

\noindent Ideal branching strategy for a project where there is parallel work is to have a local branch for each member in the team (in his/hers workspace) in order to isolate changes. The reader may ask, why is it necessary to isolate changes? The answer is that isolation of changes is needed in order for each developer to choose when he/she shall be struck by the other developers changes \cite{Appleton}. This will give the developer time to prepare for whatever impact the other developers changes will have on his/her changes.

\noindent Now some attentive readers may scream "this will introduce the simultaneous update problem!"\cite{Babich}. It is correct that the simultaneous update problem will be present with such a solution, but it is our SCM tools duty to detect these conflicts. The last person to integrate will be the one who will have to solve the merge conflicts, which will indirectly motivate the developers to integrate as often as possible.

\noindent Ideally, one branch for every release is recommended to reach optimum reproducibility of problems that customers may have \cite{Vance}. This is a necessity, because the customer may use old hardware, which is not capable of running the latest version of the software.

\noindent Appleton \cite{Appleton} has mentioned two different branching styles ("Early Branching" and "Deferred Branching"). Early branching is the more safe approach, whereas deferred branching provides more liveness\cite{Appleton}, which means more time-effective work at the trade-off of safety. In PonteVeccios case, adopting early branching would be a safe route, but is it optimal? Deferred branching on the other hand means less overheads, less merging, less isolation. Are these benefits worth it at the cost of less safety? 

\noindent{}In PonteVeccios case, we suggest them to adopt early branching, on the basis that they are developing a distributed system, which means that reproducibility is already hard to achieve \cite{Bays}. The extra safety in early branching will increase the already unfavourable odds of reproducibility, which could be critical in distributed systems such as PonteVeccio's. As mentioned earlier, a customer may not be able to run the latest software version due to certain limitations, which means that support for old versions is critical. This is why we suggest PonteVeccio to take the safe route (early branching) instead of the hasty route (deferred branching).

\subsection{Recommended Merging style/JL}
\noindent According to Appleton \cite{Appleton} the choice of merging style often (but not always) follows from the chosen branching style.

\noindent We recommend continuous merging, that means merging whenever a task is completed. Therefore we recommend the merging pattern "Merge Early and often" \cite{Appleton}. A relatively small company like PonteVecchio would have no problems to merge changes as soon as they are done, therefore this model would be optimal. Too many changes at once would mean that the development has to be halted in order to integrate the changes. Everyone that has taken the SCM course will probably agree that the merging is too infrequent if programmers can be given a week off while others are integrating the changes.

\subsection{How to handle variants?/JL}
Since the application that PonteVeccio develops has a mobile version and a desktop version, it is inevitable to not have any variants in the company. The main question in this subtopic is how are we going to handle these variants in the best possible way?

\noindent The two main variant management methods that we debated could be of use to the project was the variant segregation method and the single source variation method \cite{Mahler}. Our final choice fell on the variant segregation model due to the following reasons:
\begin{itemize}
\item PonteVeccio is a fairly small company, but the project is still enormous for one person to overview. Using the single source variation variant method would probably stall the developers often, due to the drawback that the code is harder to read \cite{Mahler}.
\item The risk of redundancy (also known as the double maintenance problem\cite{Babich}) is the main drawback of the variant segregation model\cite{Mahler}. The impact of this drawback is not a big concern, since we only have two variants (desktop and mobile). It would be an entirely different story if we had hundreds of variants. Since the common functionality is located outside the variants, one may get the idea to just fix the bug reports that were reported in each separate version, and ignore it in the other until someone reports it there as well. The conclusion is that we would not recommend such a strategy for PonteVeccio, otherwise the mobile and the desktop version that were intended to be similar to each other, will drift too far apart in terms of functionality. We therefore recommend PonteVeccio to live with the double maintenance problem in this case.

\end{itemize}
\noindent
\subsection{When to make a baseline?/JL}
We recommend PonteVeccio to establish baselines whenever there are enough change requests to make it worthwhile to establish a new baseline. If for example a critical fix is needed, one shall not be foolish and establish a new baseline just for that purpose. Instead, one should fix the issue as soon as possible and not waste time on other things. 

\noindent A baseline should follow the baseline reproducibility principle \cite{Appleton2}, which means that a baseline must be reproducible. This indicates that it may be a good idea to create a new baseline for every new release. A new baseline for every new release may not be needed, but since a release should be reproducible, a baseline following the reproducibility principle can be good as documentation when comparing releases.

\noindent Separate baselines for variants may sometimes be needed. In PonteVeccios case, we recommend separate baselines for each variant, as the software has a mobile version and a desktop version. The mobile version may have a higher emphasis on being as resource efficient as possible, whereas the desktop version may have other priorities \cite{Nielsen}.

\subsection{How to accomplish reproducibility?/JL}
\noindent Since the application that PonteVeccio is developing is an Enterprise scheduling app, with on-line functionality, it is a distributed system. In distributed systems, it is unfortunately not easy to ensure reproducibility \cite{Bays}, as some dependencies throughout the network may change. We can however put the reproducibility odds to our advantage by by saving all the released file versions, the baselines and the documents associated with each release. 

\section{SCM tool selection/JK}
\todo{Discuss, discuss, discuss}
Selecting the right SCM tool is a rather complicated process that should not be underestimated, the tool will be used for a very long period of time and a well evaluated tool selection process will increase the chances of finding a tool suited for the company's needs. There already exists general checklists of how the selection process could look like and the following subsections will give a brief summary of the most fundamental tasks during this process rather than any concrete tool selection. As Schamp describes it\cite{Schamp}:
"You cannot be concrete until you have evaluated the environment and the process". For example The Evaluation of Configuration Management Version Control Tools, \cite{ABB} describes some guidelines to follow while evaluating SCM tools.

\subsection{Tool selecting process}
Before the tool selection process is ready to be carried out the organization must be willing to accept and deploy it. If the organization lacks commitment for the new tool the risk of failure is imminent. Therefore in order to increase the chances for the process to success it's important to make sure that the organization is mature enough by proving its knowledge about \todo{Gör tydligare vad punkterna är..} \cite{ABB}
\begin{itemize}
\item The wanted benefit with a tool or a new tool.
\item The process used or the process needed for successful software development
\item See the future of needs in local SCM infrastructure and preocesses
\item Have the liquidity to buy and deploy the new tool or tools.
\end{itemize}
If the organization fails to prove its knowledge about these things then it probably also lacks commitment to the tool selection project which is then more likely to fail due to this.
\todo{PonteVechhio has proven all factors due to CEO commitment -> conslusion}

\subsection{Obtain organizational support and management commitment}
One of Moreiras \cite{Moreira} common success criterias is the sponsorship or management support. If we can convince management of the increased productivity thanks to the tool, then we'll also have their commitment to the implementation process which is necessary in order to adapt SCM throughout the organization \cite{Sayko}. The management attention can be obtained by highlighting the economical benefits in the long run rather than the technical ones. If we success in estimating a high return of investment to present to the management then there's a higher chance that the management will see the benefits of a proper SCM implementation and give their commitment to the tool adaption \cite{Sayko}. 

\subsection{Selecting group}
The tool selection should be carried out as its own project with a number of representatives from all different departments which are going to be involved in the new tool. The selecting group should at least consist of a sponsor or project manager, developers, testers and technical staff\cite{Sayko}. If there doesn't already exist any individuals responsible for the SCM process someone with knowledge about the company SCM policy such as release and branching strategies should be selected to be responsible for the implementation process as the SCM manager. In case there's more than one site it's a good idea to also appoint a SCM manager for each different site to be included in the selecting group \cite{ABB}.

\subsection{Determine the scope of the effort}
To be able to calculate the return of investment we'll have to calculate an effort estimation for the implementation process. The impact of the effort will depend on the scope that the new tool will affect, this scope is likely bigger than one first think of. As sayko explains it: "For example selecting a tool primarily for version control of source code files may eventually lead to discussions about how software configuration management fits into the organization's software development process" \cite{Sayko}\todo{Diskutera citaten}. Sayko also explains that: "The cost is not merely the purchase price of the SCM tool, but the learning curve and productivity impact that occurs during the tool adoption process" \cite{Sayko}, which might not be true for small companies\todo{Move discussion to conclusion}. Another thing that will affect the effort estimation is whether there already exists a tool that needs to be replaced with the new one. In PonteVecchios case this implementation effort is relatively small since they doesn't have any tools that needs to be replaced nor does they have lot of staff that needs training in the new tool. Further the effort estimation is highly connected to the organizational needs discussed in the following subsection.

\subsection{Determine organizational needs}
One cannot simply select a SCM tool without carefully investigate the company needs.
In order to determine an appropriate level of requirements that the organization SCM process will put on the tool it's a good idea to start investigate in organizations needs for example by interviewing a representative cross section of the organization's staff such as developers, QA personnel and management and see what they expects the tool to help them with.
It's rather important to not only make sure that the SCM process is supported by the tool but also that it does so without being too complex \cite{ABB}\todo{Diskutera vad som är för komplext}.

PonteVecchio probably desires a tool that helps them with the complete SCM process rather than just the version control features and further also facing a more elaborated tool evaluation and adoption process. If you only focus on tools supported for your SCM process, for example agile development, then there's directly a smaller spectra of tools to choose from which will ease the selection process \cite{Sayko}.

\subsection{Desirable properties}
One should not underestimate the number of requirements that will be required by the company's SCM process and will further also need to be supported by the tool. The following properties are some of the requirements 
\begin{itemize}
\item SCM strategies (long transaction, composition model, change set model)
\item SCM process support (branching, releases)
\item Freeze code (baseline handling, file locks)
\item Branching strategies \& Merging (tracking between merges)
\item History and diff functions
\item Distributed development
\item Support for remote sites
\item Integration to other tools such as IDEs
\item Scriptable to enable automation of tasks
\item Status accounting (developer requirements)
\item Ease of use
\item Fast setup of Workspace
\item Daily update to repository
\item Working offline or when network not available
\item Performance
\item Installation and easy deployment
\item Platforms required
\item Support for remote sites (repository)
\item Support for distributed revision control or DVCS
\item Client/server architecture
\item Security (Backup)
\item Vendor support
\item Modifiability
\item Lifespan
\item Support for identification and traceability
\end{itemize}
There are many different properties that have do be taken into account and by using the following guidelines during the process of deciding which properties are the most important ones will be helpful.
Some of these properties are found in a evaluation list by Schamp \cite{Schamp} where he put grades of importance for the different properties depending on their importance for different SCM processes.
Take help from tool vendor representatives, test drive the tools with some different scenarios both regular and irregular ones anes.

Organization are growing, need for SCM process lead to need of proper SCM tool,  according to: 
Tool selecting commitment according to paper. 
\begin{itemize}
\item The wanted benefit with a tool or a new tool..
\item The process used or the process needed for successful software development.
\item See the future of needs in local SCM infrastructure and processes.
\item Have the liquidity to buy and deploy the new tool or tools.
\end{itemize}

\subsection{Identify Candidate SCM Tools}
Make a list of tools supported by the development platform and IDE. Rank them in process support order \remove{Hur göra grovsortering?}

\subsection{Select and SCM tool}
Consider the properties support and process support and pick the one that feels best.
Must have knowledge in all SCM areas and understand the common goal for the group.
Must be able to make decitions and to do compromises since no tool is perfect. from all areas of the company not just management.
Developed use cases, scenarios

\subsection{Adapt the tool}
To avoid organization-wide chaos, begin the adoption on a small pilot project. Rather time consuming process.
Train staff, proper education sessions.
The migration process is easy since they doesn't have any history that has to be migrated.

\section{Conclusion}

\begin{thebibliography}{1}

\bibitem{Babich} Wayne A. Babich, Software Configuration Management, COORDINATION FOR TEAM PRODUCTIVITY.

\bibitem{Kelly} Kelly, Chapter 5: The CM TEAM AND LIBRARY, subchapter: RECRUITMENT PROBLEMS 

\bibitem{Kelly2} Kelly, Chapter 7: WHAT DOES AND DOES NOT NEED CONTROLLING? 

\bibitem{Moreira} Mario Moreira: The 3 Software Configuration Management Implementation Levels (1999)

\bibitem{Compton}Compton, The Software Configuration Manager, subchapter: QUALIFICATIONS

\bibitem{Koskela}Juha Koskela, Software configuration management in agile methods
\url{http://www.vtt.fi/inf/pdf/publications/2003/P514.pdf}

\bibitem{Keys} Jessica Keyes, Software Configuration Management (2004) \url{http://books.google.se/books?id=0vjMlBz4nC0C&printsec=frontcover&hl=sv&source=gbs_ge_summary_r&cad=0#v=onepage&q&f=false}

\bibitem{Appleton} Brad Appleton, Streamed Lines: Branching Patterns for Parallel Software Development (1998)

\bibitem{Appleton2} Brad Appleon, The Baseline Reproducibility Principle (2005)

\bibitem{Mahler} Alex Mahler, Variants: Keeping Things Together and Telling Them Apart (1994)

\bibitem{Borraci} Lorenzo Borraci, A Return on Investment Model for Software Configuration Management (2005)

\bibitem{Vinter} Bendix \& Vinter: Configuration Management from a developer's Perspective (2001)

\bibitem{Bays} Bays, chapter 4: Builds

\bibitem{Daniels3} Daniels, Chapter 3: Configuration Identification

\bibitem{ABB} Dag Ehnbom, Jon Hasselgren, Anders Nilsson, David Svensson: The Evaluation of Configuration Management Version Control Tools (2004)

\bibitem{Sayko} Michael Sayko: \url{http://www.cmcrossroads.com/article/practical-approach-selecting-and-adopting-scm-tool} (2006), visited 27-12-2014

\bibitem{Schamp} Schamp /JK

\bibitem{Vance} Stephen Vance, Advanced SCM Branching Strategies, \url{http://vance.com/steve/perforce/Branching_Strategies.html}, visited 31-12-2014

\bibitem{Nielsen} Jakob Nielsen, Mobile Site vs. Full Site, \url{http://www.nngroup.com/articles/mobile-site-vs-full-site/}, visited 31-12-2014

\bibitem{Merant} Merant: The Business Case for Software Configuration Management

\bibitem{iDA} Interchange of Data between Administrations (no specified author), Configuration Management Plan, \url{http://www.google.se/url?sa=t&rct=j&q=&esrc=s&source=web&cd=6&ved=0CE0QFjAF&url=http\%3A\%2F\%2Fec.europa.eu\%2Fidabc\%2Fservlets\%2FDoc4b4e.doc\%3Fid\%3D18630&ei=7-imVNOcJ-LOyQP124DgCw&usg=AFQjCNHnGmTINV3aG8xJlu_6xpr5e0CK_Q&bvm=bv.82001339,d.bGQ}, visited 02-01-2014 

\end{thebibliography}
\end{document}